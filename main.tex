\documentclass[a4paper]{article}

\usepackage[top=2.5cm]{geometry}
\usepackage[italian]{babel}
\usepackage{lmodern}
\usepackage{parskip}
\usepackage[hidelinks]{hyperref}
\usepackage{amsmath}
\usepackage{tcolorbox}
\usepackage{textcomp}
\usepackage[italiano,noline]{algorithm2e}
\usepackage{amsthm}
\usepackage{setspace}
\usepackage{graphicx}
\usepackage{amssymb}
\usepackage{cancel}
\usepackage{multirow}
\usepackage{float}
\usepackage[normalem]{ulem}
\usepackage{mathtools}
\usepackage{enumitem}
\usepackage{relsize}
\usepackage{pgfmath}
\usepackage{esint}
\usepackage{xfrac}
\usepackage{caption}
\usepackage{thmtools}
\usepackage[table,dvipsnames]{xcolor}
\usepackage{subcaption}
%\usepackage{calrsfs} %corsivo strano
\usepackage{amsfonts}
\usepackage{mathdots}
\usepackage[scr=boondoxo,scrscaled=1.05]{mathalfa}
\usepackage{tikz}
\usepackage{bm}

\newcommand{\quotes}[1]{``#1''}
\newcommand{\cantor}[1]{\ensuremath{\langle}#1\ensuremath{\rangle}}
\newcommand{\SC}{\ensuremath{\mathscr{C}}}
\newcommand{\RN}{\ensuremath{\mathbb{R}}}
\newcommand{\N}{\ensuremath{\mathbb{N}}}
\newcommand{\X}{\ensuremath{\mathcal{X}}}
\newcommand{\Z}{\ensuremath{\mathbb{Z}}}
\newcommand{\dx}{\ensuremath{\text{des}}}
\newcommand{\sx}{\ensuremath{\text{sin}}}
\newcommand{\RAM}{\ensuremath{\text{RAM}}}
\newcommand{\PROG}{\ensuremath{\text{PROG}}}
\newcommand{\WHILE}{\ensuremath{\text{WHILE}}}
\newcommand{\dotminus}{\mathbin{\ooalign{\hss\raise1ex\hbox{.}\hss\cr\mathsurround=0pt$-$}}}
\newcommand{\goto}[2]{
    \ensuremath{\texttt{IF}\text{ $#1=0$ }\texttt{THEN GOTO }\text{$#2$}}
}
\newcommand{\Stato}{\ensuremath{\mathscr{S}}}
\newcommand{\semRAM}{\ensuremath{{\mathlarger\varphi}}}
\newcommand{\while}[1]{\ensuremath{
    \texttt{while } #1\neq0 \texttt{ do }
}}
\newcommand{\textbegin}{\ensuremath{\texttt{begin }}}
\newcommand{\textend}{\ensuremath{\texttt{ end}}}
\newcommand{\semWHILE}{\ensuremath{{\mathlarger\psi}}}
\newcommand{\opsemWHILE}[2]{\ensuremath{{
    \rotatebox[origin=c]{180}{\textrbrackdbl}\hspace{.1em}
    #1\hspace{.1em}\text{\textrbrackdbl}(#2)
}}}
\newcommand{\opsemWHILEnoP}[1]{\ensuremath{{
    \rotatebox[origin=c]{180}{\textrbrackdbl}\hspace{.1em}
    #1\hspace{.1em}\text{\textrbrackdbl}
}}}
\newcommand{\sottolinea}[1]{\ensuremath{\text{\underline{$#1$}}}}
\newcommand{\elem}{\ensuremath{\text{ELEM}}}
\newcommand{\ricPar}{\ensuremath{\mathscr{P}}}
\newcommand{\comp}{\ensuremath{\text{COMP}}}
\newcommand{\RP}{\ensuremath{\text{RP}}}
\newcommand{\ricPrim}{\ensuremath{\text{RICPRIM}}}
\newcommand{\MIN}{\ensuremath{\text{MIN}}}
\newcommand{\Pro}{\ensuremath{\text{Pro}}}
\newcommand{\ricTot}{\ensuremath{\mathcal{T}}}
\newcommand{\sistProg}{\ensuremath{\{\semRAM_i\}}}
\newcommand{\sistProgW}{\ensuremath{\{\semWHILE_j\}}}
\newcommand{\problema}[3]{
    \begin{tcolorbox}[
        colback=white,
        sharp corners,
        boxrule=.3mm,
        boxsep=0pt,
        left=6pt,
        hbox
    ]
        \begin{tabular}{r l}
            \multicolumn{2}{l}{#1}\\
            Istanza:& #2\\
            Domanda:& #3 ?
        \end{tabular}
    \end{tcolorbox}
}
\newcommand{\fsol}[1][\Pi]{\ensuremath{\Phi_{#1}}}
\newcommand{\termina}{\ensuremath{\textup{\text{\textdownarrow}}}}
\newcommand{\ntermina}{\ensuremath{\textup{\text{\textuparrow}}}}
\newcommand{\vd}{\ensuremath{\vdots}}

\newtheorem{theorem}{Teorema}
\newtheorem{corollario}{Corollario}

\setlist[description]{labelindent=0pt}

\begin{document}

\title{
    \includegraphics[width=\textwidth]{Mere.png}\\
    \Huge Informatica Teorica\\[.2em]
    \large Prof. Mereghetti\\[.2em]
    \Large Anno accademico 2023/24
}
\author{
    \large Note di Mauro Tellaroli
}
\date{}

\maketitle

\thispagestyle{empty}

\clearpage

\addtocounter{page}{-1}
\tableofcontents
\clearpage

\setcounter{section}{-1}
\section{Introduzione}
L'informatica è la disciplina che studia l'informazione e la sua 
elaborazione \textbf{automatica}. L'elaborazione in questione non
è legata a nessun mezzo, si tratta quindi di una qualsiasi elaborazione
che può avvenire con o senza un computer.

Obiettivo di questo corso è rispondere a due domande:

\begin{enumerate}
    \item Cosa è calcolabile automaticamente? $\rightarrow$
        Teoria della calcolabilità
    \item Quanto \quotes{costa} risolvere un problema? $\rightarrow$
        Teoria della complessità
\end{enumerate}
\clearpage

\section{Prerequisiti matematici}

\subsection*{Classi di funzioni $f:A\rightarrow B$}
\subsubsection*{Iniettive}
$ f \text{ è iniettiva se } \forall a_1,a_2 \in A: a_1\neq a_2 \
\Rightarrow \ f(a_1)\neq f(a_2) $
\subsubsection*{Suriettive}
$ f \text{ è suriettiva se } \forall b\in B \ \ \exists a\in A : f(a)=b $
\subsubsection*{Biettive}
$f$ è biettiva se è sia iniettiva che suriettiva.

\subsection*{Composizione di funzioni}
Date $f:A\rightarrow B$ e $g:B\rightarrow C$, si definisce $f$ composto $g$
come la funzione $g\circ f:A\rightarrow C$ come:
$$ g\circ f(a) = g(f(a)) $$
La composizione non è un operatore commutativo.

\subsection*{Funzioni parziali e totali}
La notazione $f(a)$\textdownarrow \ indica che la funzione è definita su $a$, ovvero
che esiste un valore $b$ del codominio tale che $f(a)=b$.

Al contrario, la notazione $f(a)$\textuparrow \ indica che la funzione \textbf{non} è
definita su $a$.

Una funzione $f:A\rightarrow B$ definita su tutto il suo dominio è detta totale. Se 
invece esistono dei valori del dominio nei quali $f$ non è definita, $f$ è detta 
parziale:
$$ f \text{ è \textbf{totale} se } \forall a\in A \ \ f(a)\text{\textdownarrow} $$
$$ f \text{ è \textbf{parziale} se } \exists a\in A : f(a)\text{\textuparrow} $$

\subsubsection*{Campo di esistenza}
Dalla definizione di funzione parziale si intuisce come l'insieme di tutti i valori
nel quale la funzione $f:A\rightarrow B$ è definita, non sempre coincide con il dominio
$A$. Questo insieme è detto \textbf{campo di esistenza di $f$} e si denota con $Dom_f$:
$$ Dom_f = \{ a\in A: f(a)\text{\textdownarrow} \} \subseteq  A $$

\subsubsection*{Totalizzazione di una funzione parziale}
Presa una funzione $f:A\rightarrow B$ parziale, la si può totalizzare, ovvero rendere
totale, aggiungendo al codominio un valore $\perp$ che rappresenta il caso indefinito:
$$ f:A\rightarrow B \ \underset{\text{totalizzazione}}{\longrightarrow} \ 
f:A\rightarrow B \cup \{\perp\}$$
$$ f(a) = \begin{cases}
f(a) & a\in Dom_f\\
\perp & \text{altrimenti}\\
\end{cases} $$
L'insieme $B \cup \{\perp\}$ viene abbreviato con $B_\perp$.

\subsection*{Prodotto cartesiano}
$$ A\times B = \{(a,b):a\in A \wedge b\in B\} $$
L'operatore $\times$ non gode della proprietà commutativa.
$$ \underbrace{A\times A\times\dots\times A}_{n \text{ volte}} = A^n $$

\subsection*{Insiemi di funzioni}
Tutte le funzioni che vanno da $A$ a $B$ è detto $B^A$:
$$ B^A = \{f:A\rightarrow B\} $$
$$ B^A_\perp = \{f:A\rightarrow B_\perp\} $$

\subsection*{Funzione di valutazione}
Si definisce funzione di valutazione $\omega:B_\perp^A\times A \rightarrow B$ con:
$$ w(f,a) = f(a) $$
\begin{itemize}
    \item Fissando $a$ provo tutte le funzioni su $a$;
    \item Fissando $f$ ottengo il suo grafico.
\end{itemize}

\subsection*{Relazione binaria}
Si definisce relazione binaria $R$ sull'insieme $A$, un elenco di coppie ordinate
di elementi di $A$: $R\subseteq A^2$. Due elementi $a,b\in A$ sono in relazione 
$R$ se $(a,b)\in R$. Si usa la notazione:
\begin{itemize}
    \item $a \ R \ b$: $a$ è in relazione $R$ con $b$;
    \item $a\ \cancel{R} \ b$: $a$ non è in relazione $R$ con $b$;
\end{itemize}

\subsubsection*{Relazione di equivalenza}
$R\subseteq A^2$ è una relazione di equivalenza se gode di:
\begin{enumerate}
    \item Riflessività: $\forall a \in A \quad a \ R \ a$
    \item Simmetria: $\forall a,b \in A \quad a \ R \ b \ \Leftrightarrow \ b \ R \ a$
    \item Transitività: $\forall a,b,c \in A \quad a \ R \ b
    \ \wedge \ b \ R \ c \Rightarrow a \ R \ c $
\end{enumerate}

\subsubsection*{Classe di equivalenza}
Si definisce classe di equivalenza $[a]_R$ l'insieme degli elementi in relazione $R$
con $a$:
$$ [a]_R =\{b\in A: a \ R \ b\} $$

Tutte le classi di equivalenza di $R$ formano una partizione di $A$. L'insieme $A$
partizionato attraverso le classi di equivalenza di $R$ è detto \textbf{quoziente}
di $A$ rispetto a $R$ ed è denotato da $\sfrac{A}{R}$.

\subsubsection*{Esempio}
Si consideri la relazione $\equiv_4\subseteq\N^2$ di equivalenza modulo 4. Due numeri
sono in relazione di equivalenza modulo 4 se il resto della divisione per 4 è uguale
per entrambi.
$$ 5\equiv_4 9\ , \ 10\equiv_4 2 \ , \ \dots $$

Le classi di equivalenza sono:
\begin{align}
    [0]_4&=\{4k\}\tag{Multipli di 4}\\
    [1]_4&=\{4k+1\}\tag{Resto 1}\\
    [2]_4&=\{4k+2\}\tag{Resto 2}\\
    [3]_4&=\{4k+3\}\tag{Resto 3}
\end{align}
L'insieme $\{[0]_4,[1]_4,[2]_4,[3]_4\}=\sfrac{\N}{\equiv_4}$ è una partizione di $\N$.

\subsection*{Chiusura di insiemi rispetto ad operazioni}
Dato un insieme $U$, si definisce operazione su $U$ una qualunque funzione
$$ op: \underbrace{U\times\dots\times U}_{\displaystyle k=\text{arietà}}\to U $$

L'insieme $A\subseteq U$ è chiuso rispetto all'operazione $op:U^k\to U$ se:
$$ \forall a_1,\dots,a_k \in A : op(a_1,\dots,a_k)\in A $$

Sia $\Omega=\{op_1,\dots,op_t\}$ un insieme di operazioni su $U$, allora $A\subseteq U$
è chiuso rispetto a $\Omega$ se $A$ è chiuso per ogni $op\in\Omega$.

\subsubsection*{Esempi}
Sia $\Omega=\{+,*\}$:
\begin{itemize}
    \item PARI $\subseteq\N$ è chiuso per $\Omega$? Sì, la somma o la moltiplicazione di
        numeri parì è un numero pari.
    \item DISPARI $\subseteq\N$ è chiuso per $\Omega$? No, un controesempio è 
        $3+7=10\notin\text{DISPARI}$.
\end{itemize}

\subsection*{Chiusura di un insieme}

\begin{minipage}{.4\textwidth}
    Sia $A\subseteq U$ e $op:U^k\to U$. Si vuole trovare {\color{red}il più piccolo}
    sottoinsieme di $U$ che: \vspace{.2cm}
    \begin{enumerate}
        \setlength\itemsep{.5em}
        \item {\color{blue}Contiene $A$}
        \item {\color{blue}È chiuso per $op$}
    \end{enumerate}
\end{minipage}
\begin{minipage}{.55\textwidth}
    \begin{center}
        \begin{tikzpicture}
    \node[scale=1.4] at(2.45,1.45) {$U$};
    \draw[thick] (0,0) ellipse (3 and 1.7);
    \draw[dashed] (0,0) ellipse (1.6 and 1);
    \def\x{1.6}
    \draw[->,thick,blue] (.8,0) -- (\x-.1,0);
    \draw[->,thick,red] (3,0) -- (\x+.1,0);
    \draw[thick] (0,0) ellipse (.8 and .4) node {$A$};
\end{tikzpicture}
    \end{center}
\end{minipage}
\vspace{.3cm}

\begin{itemize}
    \item Sicuramente $U$ soddisfa le due condizioni ma {\color{red}si sta cercando
        l'insieme più piccolo};
    \item Se $A$ è chiuso per $op$ allora l'insieme cercato sarebbe $A$ stesso;
    \item Se $A$ non è chiuso per $op$ allora devo {\color{blue} allargare la ricerca ad 
        un insieme più grande}.
\end{itemize}
\vspace{.5em}
\begin{theorem}[Chiusura di un insieme]
    Sia $A\subseteq U$ e $op:U^k\to U$. Il più piccolo sottoinsieme di $U$ contenente $A$
    e chiuso rispetto a $op$ si ottiene calcolando la chiusura di $A$ rispetto a $op$, ovvero
    l'insieme $A^{op}$ definito induttivamente come:
    \begin{enumerate}
        \item $\forall a\in A \ \Rightarrow \ a\in A^{op}$
        \item $\forall a_1,\dots,a_k\in A^{op} \ \Rightarrow \ op(a_1,\dots,a_k)\in A^{op}$
        \item Nient'altro sta in $A^{op}$
    \end{enumerate}
\end{theorem}

Una versione \quotes{più operativa} per trovare $A^{op}$ è:
\begin{enumerate}
    \item $A^{op} = A$
    \item Calcola $op(a_1,\dots,a_k)=r$ su una $k$-tupla di $A$
    \item Se $r\notin A$ allora $A^{op} = A^{op}\cup\{r\}$
    \item Ripeti il punto 2 con un'altra $k$-tupla fino ad averle provate tutte
\end{enumerate}
\clearpage
\section{Teoria della calcolabilità}

\subsection{Sistema di calcolo \texorpdfstring{$\C$}{C}}
Si vuole modellare matematicamente un calcolatore o sistema di calcolo $\C$:
\vspace{.4cm}
\begin{figure}[h]
    \centering
    \begin{tikzpicture}

    \draw[->] (2.3,1.75) node[left]{$x\in$\ DATI} -- (2.9,1.75);
    \draw[->] (2.3,1.25) node[left]{$P\in$\ PROG} -- (2.9,1.25);
    \draw (3,2) rectangle (4,1);
    \node at (3.45,1.5) {$\SC$};
    \draw[->] (4.1,1.5) -- (4.7,1.5)
        node [right] {$y\ /\perp$};

\end{tikzpicture}
\end{figure}

La figura mostra il sistema di calcolo $\C$ che, preso un programma $P$
su input $x$, restituisce in output il risultato $y$ o il valore $\perp$
se il programma va in loop.

DATI è l'insieme di tutti i possibili dati di input e PROG l'insieme di
tutti i possibili programmi.

Il sistema di calcolo $\C$ non fa altro che eseguire il programma $P$ su input
$x$ ricavandone il risultato $y$:
\begin{equation}\label{eq:sistema_calcolo}
\C:\text{PROG}\times\text{DATI}\rightarrow\text{DATI}_\perp
\end{equation}

Quello che fa il programma $P$ è trasformare il dato di input $x$ in un dato
di output $y$; si può quindi dire che un programma non è altro che
una funzione che agisce da DATI in DATI:
$$ P:\text{DATI} \rightarrow \text{DATI}_\perp $$
$$ \Downarrow $$
\begin{equation}\label{eq:prog_dati} \text{PROG} =
\text{DATI}^{\text{DATI}}_\perp \end{equation}

La funzione associata al programma $P$ è detta \textbf{semantica di $P$}.

Da (\ref{eq:sistema_calcolo}) e (\ref{eq:prog_dati}) si ottiene che:
$$ \C:\text{DATI}^{\text{DATI}}_\perp\times\text{DATI}
\rightarrow\text{DATI}_\perp $$

$\C$ è una funzione di valutazione; $\C(P,x)$ è infatti la semantica di $P$.

\subsection{Potenza computazionale di \texorpdfstring{$\C$}{C}
\label{sec:pot_comp}}
Si definisce potenza computazionale di $\C$:
$$ F(\C) = \{\C(P,\_):P\in\text{PROG}\} \subseteq \text{DATI}^{\text{DATI}}_\perp $$
\textbf{$F(\C)$ contiene tutto ciò che un qualsiasi sistema di calcolo $\C$
può calcolare}. Quindi, per stabilire cosa l'informatica può risolvere, basta stabilire
il carattere dell'inclusione:
\begin{itemize}
    \item $F(\C)\subset\text{DATI}^{\text{DATI}}_\perp \Rightarrow $
        esistono problemi che l'informatica non può risolvere;
    \item $F(\C)=\text{DATI}^{\text{DATI}}_\perp \Rightarrow $
        l'informatica può risolvere tutto.
\end{itemize}

\subsection{Cardinalità di insiemi infiniti}
Per riuscire a capire se l'inclusione
$F(\C)\subseteq\text{DATI}^{\text{DATI}}_\perp$ sia propria o meno, si confronterà
la cardinalità dei due insiemi. Infatti dalla cardinalità si può ricavare che:
\begin{itemize}
    \item Se $|F(\C)|<\left|\text{DATI}^{\text{DATI}}_\perp\right|
    \quad \Rightarrow \quad F(\C)\subset\text{DATI}^{\text{DATI}}_\perp$;
    \item Se $|F(\C)|=\left|\text{DATI}^{\text{DATI}}_\perp\right|
    \quad \Rightarrow \quad F(\C)=\text{DATI}^{\text{DATI}}_\perp$.
\end{itemize}

Il concetto di cardinalità è semplice quando si tratta di insiemi finiti: basta
contare il numero di elementi che compongono l'insieme. Tuttavia, in presenza
di insiemi infiniti le cose si complicano.

Per esempio, si confrontino $\N$ e $\RN$: entrambi hanno cardinalità infinita
($|\N|=|\RN|=\infty$) eppure $\N\subset\RN$! Per comprendere quindi meglio
la cardinalità di insiemi infiniti si dovrà andare più nel dettaglio.

\subsubsection{Relazione binaria}
Si definisce relazione binaria $R$ sull'insieme $A$, un elenco di coppie ordinate
di elementi di $A$: $R\subseteq A^2$. Due elementi $a,b\in A$ sono in relazione 
$R$ se $(a,b)\in R$. Si usa la notazione:
\begin{itemize}
    \item $a \ R \ b$: $a$ è in relazione $R$ con $b$;
    \item $a\ \cancel{R} \ b$: $a$ non è in relazione $R$ con $b$;
\end{itemize}

\subsubsection{Relazione di equivalenza}
$R\subseteq A^2$ è una relazione di equivalenza se gode di:
\begin{enumerate}
    \item Riflessività: $\forall a \in A \quad a \ R \ a$
    \item Simmetria: $\forall a,b \in A \quad a \ R \ b \ \Leftrightarrow \ b \ R \ a$
    \item Transitività: $\forall a,b,c \in A \quad a \ R \ b
    \ \wedge \ b \ R \ c \Rightarrow a \ R \ c $
\end{enumerate}

\subsubsection{Classe di equivalenza}
Si definisce classe di equivalenza $[a]_R$ l'insieme degli elementi in relazione $R$
con $a$:
$$ [a]_R =\{b\in A: a \ R \ b\} $$

Tutte le classi di equivalenza di $R$ formano una partizione di $A$. L'insieme $A$
partizionato attraverso le classi di equivalenza di $R$ è detto \textbf{quoziente}
di $A$ rispetto a $R$ ed è denotato da $A/R$.

\subsubsection*{Esempio}
Si consideri la relazione $\equiv_4\subseteq\N^2$ di equivalenza modulo 4. Due numeri
sono in relazione di equivalenza modulo 4 se il resto della divisione per 4 è uguale
per entrambi.
$$ 5\equiv_4 9\ , \ 10\equiv_4 2 \ , \ \dots $$

Le classi di equivalenza sono:
\begin{align}
    [0]_4&=\{4k\}\tag{Multipli di 4}\\
    [1]_4&=\{4k+1\}\tag{Resto 1}\\
    [2]_4&=\{4k+2\}\tag{Resto 2}\\
    [3]_4&=\{4k+3\}\tag{Resto 3}
\end{align}
L'insieme $\{[0]_4,[1]_4,[2]_4,[3]_4\}=\N/\equiv_4$ è una partizione di $\N$.

\subsubsection{Insiemi isomorfi}
Due insiemi $A$ e $B$ sono \textbf{isomorfi} (o equinumerosi) se esiste una funzione
biettiva tra essi. Formalmente si indica con:
$$ A\sim B $$
La relazione di isomorfismo $\sim$ è una relazione di equivalenza in quanto:
\begin{enumerate}
    \item Riflessiva: si usi la funzione identità;
    \item Simmetrica: se esiste una funzione biettiva allora anche la sua inversa
        è biettiva;
    \item Transitiva: la composizione di due funzioni biettive è una funzione biettiva.
\end{enumerate}

Sia $\mathcal{U}$ l'insieme universo, ovvero l'insieme che contiene tutti gli insiemi.
Il quoziente di $\mathcal{U}$ rispetto a $\sim$ ($\mathcal{U}/\sim$) definisce il 
concetto di cardinalità:

\begin{figure}[H]
    \centering
    \begin{tikzpicture}[scale=1.5]

    \node at (-2,1) {$\sfrac{\mathscr{U}}{\sim}$};

    \draw[thick] (0,0) node{...} ellipse (2 and 1);
    \draw (-1.25,-.78) -- (-1.25,.78);
    \draw (-.5,-.97) -- (-.5,.97);
    \draw (.5,-.97) -- (.5,.97);
    \draw (1.25,-.78) -- (1.25,.78);

    \draw[fill,blue] (-.875,0) circle (.02);
    \draw[->,blue] (-.875,0) -- (-1.75,-1);
    \node[blue,below] at (-1.75,-1) {Classe di equivalenza};

\end{tikzpicture}
\end{figure}

Ogni partizione di $\mathcal{U}/\sim$ contiene gli insiemi tra loro isomorfi, ovvero
che hanno la stessa cardinalità.

\subsubsection*{Insiemi finiti}
Si definisca la famiglia di insiemi: 
$$J_n=\begin{cases}
\cancel{O} & n=0\\
\{1,\dots ,n\} & n>0
\end{cases}$$
$$ J_0=\{\}\ , \ J_1=\{1\} \ , \ J_{2}=\{1,2\} \ , \ J_{3}=\{1,2,3\}\ , \ \dots $$

Un'insieme $A$ ha cardinalità finita se $\exists n\in\N : A\sim J_n$ e si può dire che
$|A|=n$.

\subsubsection*{Insiemi infiniti}
Un insieme che non è finito ha cardinalità infinita.

\subsubsection{Insiemi numerabili}
Un insieme $A$ è numerabile se $\N\sim A$ (ovvero $A\in [\N]_\sim$). Vuole quindi dire
che esiste una biezione $f:\N\rightarrow A$ che permette di listare $A$ come:
$$ A = \{f(0),f(1),f(2),\dots\} $$
senza tralasciare nessun elemento.
\subsubsection*{Esempi}
\begin{tabular}{r l}
    PARI :& $f(n)=2n$ \\
    DISPARI :& $f(n)=2n+1$ \\
    $\mathbb{Z}$ :& mappo i pari nei non-negativi e i dispari nei negativi \\
    $\{0\}\cup 1\{0,1\}^*$ :& converto da binario a decimale \\
\end{tabular}

\subsubsection{Insiemi non numerabili}
Gli insiemi non numerabili sono insiemi a cardinalità infinita ma non listabili come
$\N$ (sono \quotes{più fitti}). Il re di questi insiemi è $\RN$.

\begin{theorem}
    $\RN$ è un insieme non numerabile: $$ \N \nsim \RN $$
\end{theorem}
\begin{proof} Per dimostrarlo dimostro che:
    \begin{enumerate}
        \item $\RN \sim (0,1)$: la biezione è rappresentata graficamente in figura:
            \vspace{-.2cm}
            \begin{figure}[H]
                \centering
                \begin{tikzpicture}[scale=2]

    \draw (-1.7,0) -- (1.7,0);
    \draw[densely dashed] (-1.7,0) -- (-2,0);
    \draw[densely dashed] (1.7,0) -- (2,0);
    \draw (0,.05) -- (0,-.05) node[below] {0};
    
    \draw[cyan] (1,1.5) arc[start angle=0, end angle=-180,radius=1];

    \draw (-1,1.5) -- (1,1.5);
    \draw (-1,1.55) -- (-1,1.45) node[left] {0};
    \draw (1,1.55) -- (1,1.45) node[right] {1};

    \draw[densely dashed, cyan] (-.515,.64) --  (0,1.5);
    \draw[red] (-.9,0) -- (-.515,.64) -- (-.515,1.5);
    \draw (-.515,1.45) -- (-.515,1.55) node[above] {$f(a)$};
    \draw (-.9,.05) -- (-.9,-.05) node[below] {$a$};

    \draw[densely dashed, cyan] (.68,.77) -- (0,1.5);
    \draw[red] (1.4,0) -- (.68,.77) -- (.68,1.5);
    \draw (.68,1.45) -- (.68,1.55) node[above] {$f(b)$};
    \draw (1.4,.05) -- (1.4,-.05) node[below] {$b$};

    \draw [fill] (0,1.5) circle (.02);

\end{tikzpicture}
            \end{figure}\vspace{-.6cm}
            (In realtà $\RN$ è isomorfo a un suo qualsiasi intervallo).
        \item $\N \nsim (0,1)$: dimostrazione per assurdo: assumo che $\N \sim (0,1)$;
            Questo vorrebbe dire che tutti i numeri compresi tra 0 e 1 sono numerabili.
            Elenco tutti i numeri associandoli a un numero naturale:

            \begin{minipage}{.45\textwidth}
                $$0\ \mapsto \ 0.{\color{red}a_{00}}\ a_{01}\ a_{02}\ a_{03}\ a_{04}\ \dots$$
                $$1\ \mapsto \ 0.a_{10}\ {\color{red}a_{11}}\ a_{12}\ a_{13}\ a_{14}\ \dots$$
                $$2\ \mapsto \ 0.a_{20}\ a_{21}\ {\color{red}a_{22}}\ a_{23}\ a_{24}\ \dots$$
                $$3\ \mapsto \ 0.a_{30}\ a_{31}\ a_{32}\ {\color{red}a_{33}}\ a_{34}\ \dots$$
                $$4\ \mapsto \ 0.a_{40}\ a_{41}\ a_{42}\ a_{43}\ {\color{red}a_{44}}\ \dots$$
                $$\vdots\qquad\vdots\qquad\vdots\qquad\vdots\qquad\vdots\qquad\ddots$$
                \vspace{.4cm}
            \end{minipage}
            \begin{minipage}{.48\textwidth}
                $a_{ij}$ è la $i$-esima cifra dopo lo zero del $j$-esimo numero
                nella lista.

                Se $(0,1)$ fosse numerabile tutti i suoi numeri dovrebbero far parte
                della lista.

                Si consideri il numero:
                $$ 0.c_0c_1c_2c_3\dots $$

                con: $$ c_i = \begin{cases}
                2 & a_{ii}\neq2\\
                3 & a_{ii}=2
                \end{cases} $$

            \end{minipage}
            Chiaramente $0.c_0c_1c_2c_3\dots\in(0,1)$ ma non appare nella lista:
            \begin{itemize}
                \item Differisce dal primo numero perchè $c_0\neq a_{00}$;
                \item Differisce dal secondo numero perchè $c_1\neq a_{11}$;
                \item $\dots$
                \item Differisce da qualunque numero nella lista sulla cifra 
                    {\color{red} diagonale}.
            \end{itemize}
            Ho trovato l'assurdo quindi $\N \nsim (0,1)$ (dimostrazione per 
            diagonalizzazione).
    \end{enumerate}
    Sfruttando la transitività di $\sim$ posso si può affermare quindi che:
    $$ \RN \underset{(1)}{\sim} (0,1) \underset{(2)}{\nsim} \N \quad \Rightarrow \quad \RN \nsim \N $$
\end{proof}

Tutti gli insiemi isomorfi a $\RN$ sono detti continui. Altri insiemi non numerabili sono:
\begin{itemize}
    \item Insieme delle parti di $\N$: $2^\N = \{\text{sottoinsiemi di } \N\}$
    \item Insieme delle funzioni da $\N$ a $\N$:
        $\N^\N_\perp = \{f:\N\rightarrow\N_\perp\}$
\end{itemize}

\subsection{Cosa è calcolabile?}
Ora che il concetto di cardinalità è più chiaro, si riprenda il concetto di 
potenza computazionale di un sistema di calcolo $\C$ (paragrafo \ref{sec:pot_comp}):
$$
F(\C) = \{\C(P,\_):P\in\text{PROG}\} \subseteq \text{DATI}^{\text{DATI}}_\perp
$$

Per definizione $F(\C)$ ha la stessa numerosità di PROG:
$$ F(\C) \sim \text{PROG} $$

Ragionevolmente, \textbf{ma non formalmente}, si può notare che:
\begin{itemize}
    \item PROG$\sim\N$: si prenda la stringa binaria con la quale il programma è
        salvato sul disco e si converta da binario a decimale;
    \item DATI$\sim\N$: si applichi lo stesso ragionamento del punto precedente.
\end{itemize}
Ne segue che:
$$ F(\C) \sim \text{PROG} \sim \N $$

\clearpage


\end{document}
