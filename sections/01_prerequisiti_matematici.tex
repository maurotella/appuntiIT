
\section{Prerequisiti matematici}

\subsection*{Classi di funzioni $f:A\rightarrow B$}
\subsubsection*{Iniettive}
$ f \text{ è iniettiva se } \forall a_1,a_2 \in A: a_1\neq a_2
\Rightarrow f(a_1)\neq f(a_2) $
\subsubsection*{Suriettive}
$ f \text{ è suriettiva se } \forall b\in B \ \ \exists a\in A : f(a)=b $
\subsubsection*{Biettive}
$f$ è biettiva se è sia iniettiva che suriettiva.

\subsection*{Composizione di funzioni}
Date $f:A\rightarrow B$ e $g:B\rightarrow C$, si definisce $f$ composto $g$
come la funzione $g\circ f:A\rightarrow C$ come:
$$ g\circ f(a) = g(f(a)) $$
La composizione non è un operatore commutativo.

\subsection*{Funzioni parziali e totali}
La notazione $f(a)$\textdownarrow \ indica che la funzione è definita su $a$, ovvero
che esiste un valore $b$ del codominio tale che $f(a)=b$.

Al contrario, la notazione $f(a)$\textuparrow \ indica che la funzione \textbf{non} è
definita su $a$.

Una funzione $f:A\rightarrow B$ definita su tutto il suo dominio è detta totale. Se 
invece esistono dei valori del dominio nei quali $f$ non è definita, $f$ è detta 
parziale:
$$ f \text{ è \textbf{totale} se } \forall a\in A \ \ f(a)\text{\textdownarrow} $$
$$ f \text{ è \textbf{parziale} se } \exists a\in A : f(a)\text{\textuparrow} $$

\subsubsection*{Campo di esistenza}
Dalla definizione di funzione parziale si intuisce come l'insieme di tutti i valori
nel quale la funzione $f:A\rightarrow B$ è definita, non sempre coincide con il dominio
$A$. Questo insieme è detto \textbf{campo di esistenza di $f$} e si denota con $Dom_f$:
$$ Dom_f = \{ a\in A: f(a)\text{\textdownarrow} \} \subseteq  A $$

\subsubsection*{Totalizzazione di una funzione parziale}
Presa una funzione $f:A\rightarrow B$ parziale, la si può totalizzare, ovvero rendere
totale, aggiungendo al codominio un valore $\perp$ che rappresenta il caso indefinito:
$$ f:A\rightarrow B \ \underset{\text{totalizzazione}}{\longrightarrow} \ 
f:A\rightarrow B \cup \{\perp\}$$
$$ f(a) = \begin{cases}
f(a) & a\in Dom_f\\
\perp & \text{altrimenti}\\
\end{cases} $$
L'insieme $B \cup \{\perp\}$ viene abbreviato con $B_\perp$.

\subsection*{Prodotto cartesiano}
$$ A\times B = \{(a,b):a\in A \wedge b\in B\} $$
L'operatore $\times$ non gode della proprietà commutativa.
$$ \underbrace{A\times A\times\dots\times A}_{n \text{ volte}} = A^n $$

\subsection*{Insiemi di funzioni}
Tutte le funzioni che vanno da $A$ a $B$ è detto $B^A$:
$$ B^A = \{f:A\rightarrow B\} $$
$$ B^A_\perp = \{f:A\rightarrow B_\perp\} $$

\subsection*{Funzione di valutazione}
Si definisce funzione di valutazione $\omega:B_\perp^A\times A \rightarrow B$ con:
$$ w(f,a) = f(a) $$
\begin{itemize}
    \item Fissando $a$ provo tutte le funzioni su $a$;
    \item Fissando $f$ ottengo il suo grafico.
\end{itemize}